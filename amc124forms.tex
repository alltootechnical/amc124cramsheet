%% LyX 2.1.3 created this file.  For more info, see http://www.lyx.org/.
%% Do not edit unless you really know what you are doing.
\documentclass[english]{article}
\usepackage[latin9]{inputenc}
\usepackage[a4paper]{geometry}
\geometry{verbose,tmargin=1in,bmargin=1in,lmargin=0.5in,rmargin=0.5in}
\usepackage{fancyhdr}
\pagestyle{fancy}
\setlength{\parskip}{\smallskipamount}
\setlength{\parindent}{0pt}
\usepackage{babel}
\usepackage{amsmath}
\usepackage{amssymb}
\usepackage{setspace}
\usepackage[unicode=true,
 bookmarks=true,bookmarksnumbered=false,bookmarksopen=false,
 breaklinks=true,pdfborder={0 0 0},backref=false,colorlinks=false]
 {hyperref}
\hypersetup{pdftitle={Discrete Math Cram Sheet},
 pdfauthor={Brian},
 pdfsubject={AMC 124}}

\makeatletter

%%%%%%%%%%%%%%%%%%%%%%%%%%%%%% LyX specific LaTeX commands.
%% Because html converters don't know tabularnewline
\providecommand{\tabularnewline}{\\}

%%%%%%%%%%%%%%%%%%%%%%%%%%%%%% User specified LaTeX commands.
\usepackage{multirow}
\usepackage{graphicx}
\usepackage{pgf,tikz}
\usepackage{mathrsfs}
\usetikzlibrary{arrows}
\usepackage{multicol}
\usepackage{array}
\usepackage{pgfplots}
\usepackage{tocloft}
\usepackage{tabularx}

\makeatother

\begin{document}

\lhead{\textsf{\textbf{Discrete Math Cram Sheet}}}


\rhead{\textsf{\href{http://alltootechnical.tk}{alltootechnical.tk}}}


\title{\textsf{\textbf{Discrete Math Cram Sheet}}}


\date{\textsf{\today{}}}

\maketitle
\begin{multicols}{2}

\tableofcontents{}

\end{multicols}\newpage{}

\begin{multicols*}{2}


\section{Propositional Logic}


\subsection{Truth Tables}

\begin{tabular}{|c|cccc|l|}
\cline{1-5} 
$p$ & T & T & F & F & \multicolumn{1}{l}{}\tabularnewline
$q$ & T & F & T & F & \multicolumn{1}{l}{}\tabularnewline
\hline 
F & F & F & F & F & contradiction\tabularnewline
$p\veebar q$ & F & F & F & T & joint denial\tabularnewline
$p\nleftarrow q$ & F & F & T & F & converse nonimplication\tabularnewline
$\lnot p$ & F & F & T & T & left negation\tabularnewline
$p\nrightarrow q$ & F & T & F & F & nonimplication\tabularnewline
$\lnot q$ & F & T & F & T & right negation\tabularnewline
$p\oplus q$ & F & T & T & F & exclusive disjunction\tabularnewline
$p\barwedge q$ & F & T & T & T & alternative denial\tabularnewline
$p\wedge q$ & T & F & F & F & conjunction\tabularnewline
$p\leftrightarrow q$ & T & F & F & T & biconditional\tabularnewline
$q$ & T & F & T & F & right projection\tabularnewline
$p\rightarrow q$ & T & F & T & T & implication\tabularnewline
$p$ & T & T & F & F & left projection\tabularnewline
$p\leftarrow q$ & T & T & F & T & converse implication\tabularnewline
$p\vee q$ & T & T & T & F & disjunction\tabularnewline
T & T & T & T & T & tautology\tabularnewline
\hline 
\end{tabular}


\subsection{Logical Equivalences}

\begin{tabular}{ll}
\hline 
\noalign{\vskip\doublerulesep}
Identity & %
\begin{minipage}[t]{0.5\columnwidth}%
\begin{spacing}{0.5}
\noindent $p\wedge\mathrm{T}\equiv p$

\noindent $p\vee\mathrm{F}\equiv p$\end{spacing}
%
\end{minipage}\tabularnewline
\hline 
\noalign{\vskip\doublerulesep}
Domination & %
\begin{minipage}[t]{0.5\columnwidth}%
\begin{spacing}{0.5}
\noindent $p\vee\mathrm{T}\equiv\mathrm{T}$

\noindent $p\wedge\mathrm{F}\equiv\mathrm{F}$\end{spacing}
%
\end{minipage}\tabularnewline
\hline 
\noalign{\vskip\doublerulesep}
Idempotent & %
\begin{minipage}[t]{0.5\columnwidth}%
\begin{spacing}{0.5}
\noindent $p\wedge p\equiv p$

\noindent $p\vee p\equiv p$\end{spacing}
%
\end{minipage}\tabularnewline
\hline 
\noalign{\vskip\doublerulesep}
Commutative & %
\begin{minipage}[t]{0.5\columnwidth}%
\begin{spacing}{0.5}
\noindent $p\wedge q\equiv q\wedge p$

\noindent $p\vee q\equiv q\vee p$\end{spacing}
%
\end{minipage}\tabularnewline
\hline 
\noalign{\vskip\doublerulesep}
Associative & %
\begin{minipage}[t]{0.5\columnwidth}%
\begin{spacing}{0.5}
\noindent $p\wedge\left(q\wedge r\right)\equiv\left(p\wedge q\right)\wedge r$

\noindent $p\vee\left(q\vee r\right)\equiv\left(p\vee q\right)\vee r$\end{spacing}
%
\end{minipage}\tabularnewline
\hline 
\noalign{\vskip\doublerulesep}
Distributive & %
\begin{minipage}[t]{0.5\columnwidth}%
\begin{spacing}{0.5}
\noindent $p\vee\left(q\wedge r\right)\equiv\left(p\vee q\right)\wedge\left(p\vee r\right)$

\noindent $p\wedge\left(q\vee r\right)\equiv\left(p\wedge q\right)\vee\left(p\wedge r\right)$\end{spacing}
%
\end{minipage}\tabularnewline
\hline 
\noalign{\vskip\doublerulesep}
De Morgan's & %
\begin{minipage}[t]{0.5\columnwidth}%
\begin{spacing}{0.5}
\noindent $\lnot\left(p\wedge q\right)\equiv\lnot p\vee\lnot q$

\noindent $\lnot\left(p\vee q\right)\equiv\lnot p\wedge\lnot q$\end{spacing}
%
\end{minipage}\tabularnewline
\hline 
\noalign{\vskip\doublerulesep}
Absorption & %
\begin{minipage}[t]{0.5\columnwidth}%
\begin{spacing}{0.5}
\noindent $p\wedge\left(p\vee q\right)\equiv p$

\noindent $p\vee\left(p\wedge q\right)\equiv p$\end{spacing}
%
\end{minipage}\tabularnewline
\hline 
\noalign{\vskip\doublerulesep}
Negation & %
\begin{minipage}[t]{0.5\columnwidth}%
\begin{spacing}{0.5}
\noindent $p\vee\lnot p\equiv\mathrm{T}$

\noindent $p\wedge\lnot p\equiv\mathrm{F}$\end{spacing}
%
\end{minipage}\tabularnewline
\hline 
\noalign{\vskip\doublerulesep}
Double Negation & %
\begin{minipage}[t]{0.5\columnwidth}%
\begin{spacing}{0.5}
\noindent $\lnot\left(\lnot p\right)\equiv p$\end{spacing}
%
\end{minipage}\tabularnewline
\hline 
\end{tabular}


\subsubsection*{Involving Biconditionals}
\begin{itemize}
\item $p\leftrightarrow q\equiv\left(p\rightarrow q\right)\wedge\left(q\rightarrow p\right)$
\item $p\leftrightarrow q\equiv\lnot p\leftrightarrow\lnot q$
\item $p\leftrightarrow q\equiv\left(p\wedge q\right)\vee\left(\lnot p\wedge\lnot q\right)$
\item $\lnot\left(p\leftrightarrow q\right)\equiv p\leftrightarrow\lnot q$
\end{itemize}

\subsubsection*{Involving Conditional Statements}
\begin{itemize}
\item $p\rightarrow q\equiv\lnot p\vee q$
\item $p\rightarrow q\equiv\lnot q\rightarrow\lnot p$
\item $p\vee q\equiv\lnot p\rightarrow q$
\item $p\wedge q\equiv\lnot\left(p\rightarrow\lnot q\right)$
\item $\left(p\rightarrow q\right)\wedge\left(p\rightarrow r\right)\equiv p\rightarrow\left(q\wedge r\right)$
\item $\left(p\rightarrow r\right)\wedge\left(q\rightarrow r\right)\equiv\left(p\vee q\right)\rightarrow r$
\item $\left(p\rightarrow q\right)\vee\left(p\rightarrow r\right)\equiv p\rightarrow\left(q\vee r\right)$
\item $\left(p\rightarrow r\right)\vee\left(q\rightarrow r\right)\equiv\left(p\wedge q\right)\rightarrow r$
\end{itemize}

\subsubsection*{Involving Quantifiers}
\begin{itemize}
\item $\lnot\left(\forall x.\;P\left(x\right)\right)\equiv\exists x.\;\lnot P\left(x\right)$
\end{itemize}

\subsection{Rules of Inference}

\begin{tabularx}{\columnwidth}{ll}
\hline 
\noalign{\vskip\doublerulesep}
Modus Ponens & $\begin{array}{l}
p\rightarrow q\\
\underline{p\quad\quad}\\
q
\end{array}$\tabularnewline[\doublerulesep]
\hline 
\noalign{\vskip\doublerulesep}
Modus Tollens & $\begin{array}{l}
\lnot q\\
\underline{p\rightarrow q}\\
\lnot p
\end{array}$\tabularnewline[\doublerulesep]
\hline 
\noalign{\vskip\doublerulesep}
Associative & $\begin{array}{l}
\underline{\left(p\vee q\right)\vee r}\\
p\vee\left(q\vee r\right)
\end{array}$\tabularnewline[\doublerulesep]
\hline 
\noalign{\vskip\doublerulesep}
Commutative & $\begin{array}{l}
\underline{p\wedge q}\\
q\wedge p
\end{array}$\tabularnewline[\doublerulesep]
\hline 
\noalign{\vskip\doublerulesep}
Biconditional & $\begin{array}{l}
p\rightarrow q\\
\underline{q\rightarrow p}\\
p\leftrightarrow q
\end{array}$\tabularnewline[\doublerulesep]
\hline 
\noalign{\vskip\doublerulesep}
Exportation & $\begin{array}{l}
\underline{\left(p\wedge q\right)\rightarrow r\;\;}\\
p\rightarrow\left(q\rightarrow r\right)
\end{array}$\tabularnewline[\doublerulesep]
\hline 
\noalign{\vskip\doublerulesep}
Contraposition & $\begin{array}{l}
\underline{p\rightarrow q\quad\;}\\
\lnot q\rightarrow\lnot p
\end{array}$\tabularnewline[\doublerulesep]
\hline 
\noalign{\vskip\doublerulesep}
Hypothetical Syllogism & $\begin{array}{l}
p\rightarrow q\\
\underline{q\rightarrow r}\\
p\rightarrow r
\end{array}$\tabularnewline[\doublerulesep]
\hline 
\noalign{\vskip\doublerulesep}
Material Implication & $\begin{array}{l}
\underline{p\rightarrow q\;}\\
\lnot p\vee q
\end{array}$\tabularnewline[\doublerulesep]
\hline 
\noalign{\vskip\doublerulesep}
Distributive & $\begin{array}{l}
\underline{\left(p\vee q\right)\wedge r\quad\quad}\;\\
\left(p\wedge r\right)\vee\left(q\wedge r\right)
\end{array}$\tabularnewline[\doublerulesep]
\hline 
\noalign{\vskip\doublerulesep}
Absorption & $\begin{array}{l}
\underline{p\rightarrow q\quad\qquad}\\
p\rightarrow\left(p\wedge q\right)
\end{array}$\tabularnewline[\doublerulesep]
\hline 
\noalign{\vskip\doublerulesep}
Disjunctive Syllogism & $\begin{array}{l}
p\vee q\\
\underline{\lnot p\quad\quad}\\
q
\end{array}$\tabularnewline[\doublerulesep]
\hline 
\noalign{\vskip\doublerulesep}
Addition & $\begin{array}{l}
\underline{p\qquad}\\
p\vee q
\end{array}$\tabularnewline[\doublerulesep]
\hline 
\noalign{\vskip\doublerulesep}
Simplification & $\begin{array}{l}
\underline{p\wedge q\;}\\
p
\end{array}$\tabularnewline[\doublerulesep]
\hline 
\noalign{\vskip\doublerulesep}
Conjunction & $\begin{array}{l}
p\\
\underline{q\quad\quad}\\
p\wedge q
\end{array}$\tabularnewline[\doublerulesep]
\hline 
\noalign{\vskip\doublerulesep}
Double Negation & $\begin{array}{l}
\underline{p\qquad}\\
\lnot\lnot p
\end{array}$\tabularnewline[\doublerulesep]
\hline 
\noalign{\vskip\doublerulesep}
Disjunctive Simplification & $\begin{array}{l}
\underline{p\vee p\;}\\
p
\end{array}$\tabularnewline[\doublerulesep]
\hline 
\noalign{\vskip\doublerulesep}
Resolution & $\begin{array}{l}
\;\;\:p\vee q\\
\underline{\lnot p\vee r}\\
q\vee r
\end{array}$\tabularnewline[\doublerulesep]
\hline 
\end{tabularx}


\subsection{Satisfiability}

A proposition is \emph{satisfiable} if some setting of the variables
makes the proposition true. For example, $P\wedge\lnot Q$ is satisfiable
because the expression is true if $P$ is true or $Q$ is false. On
the other hand, $P\wedge\lnot P$ is not satisfiable because the expression
as a whole is false for both settings of $P$.


\subsubsection*{2-SAT Problem}

(to follow...)


\section{Proofs}


\subsection{Mathematical Induction}


\subsection{Strong Induction}


\section{Recurrence Relations}


\section{Number Theory}


\subsection{Divisibility}


\subsection{Primes and GCD}


\subsubsection*{Greatest Common Divisor}

This can be defined by the following recurrence relation:

\[
\gcd\left(a,b\right)=\begin{cases}
a & \text{if }b=0\\
\gcd\left(b,a\bmod b\right) & \text{else}
\end{cases}
\]



\subsection{Modular Arithmetic}


\section{Graph Theory}


\section{Linear Algebra}


\section{Combinatorics}


\section{Probability}

\end{multicols*}
\end{document}
