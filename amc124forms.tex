%% LyX 2.1.3 created this file.  For more info, see http://www.lyx.org/.
%% Do not edit unless you really know what you are doing.
\documentclass[english]{article}
\usepackage[latin9]{inputenc}
\usepackage[a4paper]{geometry}
\geometry{verbose,tmargin=1in,bmargin=1in,lmargin=0.5in,rmargin=0.5in}
\usepackage{fancyhdr}
\pagestyle{fancy}
\setlength{\parskip}{\smallskipamount}
\setlength{\parindent}{0pt}
\usepackage{babel}
\usepackage{amsmath}
\usepackage{amssymb}
\usepackage{setspace}
\usepackage[unicode=true,
 bookmarks=true,bookmarksnumbered=false,bookmarksopen=false,
 breaklinks=true,pdfborder={0 0 0},backref=false,colorlinks=false]
 {hyperref}
\hypersetup{pdftitle={Discrete Math Cram Sheet},
 pdfauthor={Brian},
 pdfsubject={AMC 124}}

\makeatletter

%%%%%%%%%%%%%%%%%%%%%%%%%%%%%% LyX specific LaTeX commands.
%% Because html converters don't know tabularnewline
\providecommand{\tabularnewline}{\\}

%%%%%%%%%%%%%%%%%%%%%%%%%%%%%% Textclass specific LaTeX commands.
\newenvironment{lyxlist}[1]
{\begin{list}{}
{\settowidth{\labelwidth}{#1}
 \setlength{\leftmargin}{\labelwidth}
 \addtolength{\leftmargin}{\labelsep}
 \renewcommand{\makelabel}[1]{##1\hfil}}}
{\end{list}}

%%%%%%%%%%%%%%%%%%%%%%%%%%%%%% User specified LaTeX commands.
\usepackage{multirow}
\usepackage{graphicx}
\usepackage{pgf,tikz}
\usepackage{mathrsfs}
\usetikzlibrary{arrows}
\usepackage{multicol}
\usepackage{array}
\usepackage{pgfplots}
\usepackage{tocloft}
%%\usepackage{mathpazo}

\usetikzlibrary{calc}

\tikzset{3D/.cd,
  x/.store in=\xx, x=0,
  y/.store in=\yy, y=0,
  z/.store in=\zz, z=0
}

\tikzdeclarecoordinatesystem{3D}{%
  \tikzset{3D/.cd,#1}%
  \pgfpoint{sin(\yy)*(\xx)}{-((\xx)/75)^2+(\zz)/100*(\xx)}%
}

\makeatother

\begin{document}

\lhead{\textsf{\textbf{Discrete Math Cram Sheet}}}


\rhead{\textsf{\href{http://alltootechnical.tk}{alltootechnical.tk}}}


\title{\textsf{\textbf{Discrete Math Cram Sheet}}}


\date{\textsf{\today{}}}

\maketitle
\begin{multicols*}{2}

\tableofcontents{}

\end{multicols*}\newpage{}

\begin{multicols*}{2}


\section{Propositional Logic}


\subsection{Truth Tables}

\begin{tabular}{|c|cccc|l|}
\cline{1-5} 
$p$ & T & T & F & F & \multicolumn{1}{l}{}\tabularnewline
$q$ & T & F & T & F & \multicolumn{1}{l}{}\tabularnewline
\hline 
F & F & F & F & F & contradiction\tabularnewline
$p\veebar q$ & F & F & F & T & joint denial\tabularnewline
$p\nleftarrow q$ & F & F & T & F & converse nonimplication\tabularnewline
$\lnot p$ & F & F & T & T & left negation\tabularnewline
$p\nrightarrow q$ & F & T & F & F & nonimplication\tabularnewline
$\lnot q$ & F & T & F & T & right negation\tabularnewline
$p\oplus q$ & F & T & T & F & exclusive disjunction\tabularnewline
$p\barwedge q$ & F & T & T & T & alternative denial\tabularnewline
$p\wedge q$ & T & F & F & F & conjunction\tabularnewline
$p\leftrightarrow q$ & T & F & F & T & biconditional/equivalence\tabularnewline
$q$ & T & F & T & F & right projection\tabularnewline
$p\rightarrow q$ & T & F & T & T & implication\tabularnewline
$p$ & T & T & F & F & left projection\tabularnewline
$p\leftarrow q$ & T & T & F & T & converse implication\tabularnewline
$p\vee q$ & T & T & T & F & disjunction\tabularnewline
T & T & T & T & T & tautology\tabularnewline
\hline 
\end{tabular}


\subsection{Logical Equivalences}

\begin{tabular}{ll}
\hline 
\noalign{\vskip\doublerulesep}
Identity & %
\begin{minipage}[t]{0.5\columnwidth}%
\begin{spacing}{0.5}
\noindent $p\wedge\mathrm{T}\equiv p$

\noindent $p\vee\mathrm{F}\equiv p$\end{spacing}
%
\end{minipage}\tabularnewline
\hline 
\noalign{\vskip\doublerulesep}
Domination & %
\begin{minipage}[t]{0.5\columnwidth}%
\begin{spacing}{0.5}
\noindent $p\vee\mathrm{T}\equiv\mathrm{T}$

\noindent $p\wedge\mathrm{F}\equiv\mathrm{F}$\end{spacing}
%
\end{minipage}\tabularnewline
\hline 
\noalign{\vskip\doublerulesep}
Idempotent & %
\begin{minipage}[t]{0.5\columnwidth}%
\begin{spacing}{0.5}
\noindent $p\wedge p\equiv p$

\noindent $p\vee p\equiv p$\end{spacing}
%
\end{minipage}\tabularnewline
\hline 
\noalign{\vskip\doublerulesep}
Commutative & %
\begin{minipage}[t]{0.5\columnwidth}%
\begin{spacing}{0.5}
\noindent $p\wedge q\equiv q\wedge p$

\noindent $p\vee q\equiv q\vee p$\end{spacing}
%
\end{minipage}\tabularnewline
\hline 
\noalign{\vskip\doublerulesep}
Associative & %
\begin{minipage}[t]{0.5\columnwidth}%
\begin{spacing}{0.5}
\noindent $p\wedge\left(q\wedge r\right)\equiv\left(p\wedge q\right)\wedge r$

\noindent $p\vee\left(q\vee r\right)\equiv\left(p\vee q\right)\vee r$\end{spacing}
%
\end{minipage}\tabularnewline
\hline 
\noalign{\vskip\doublerulesep}
Distributive & %
\begin{minipage}[t]{0.5\columnwidth}%
\begin{spacing}{0.5}
\noindent $p\vee\left(q\wedge r\right)\equiv\left(p\vee q\right)\wedge\left(p\vee r\right)$

\noindent $p\wedge\left(q\vee r\right)\equiv\left(p\wedge q\right)\vee\left(p\wedge r\right)$\end{spacing}
%
\end{minipage}\tabularnewline
\hline 
\noalign{\vskip\doublerulesep}
De Morgan's & %
\begin{minipage}[t]{0.5\columnwidth}%
\begin{spacing}{0.5}
\noindent $\lnot\left(p\wedge q\right)\equiv\lnot p\vee\lnot q$

\noindent $\lnot\left(p\vee q\right)\equiv\lnot p\wedge\lnot q$\end{spacing}
%
\end{minipage}\tabularnewline
\hline 
\noalign{\vskip\doublerulesep}
Absorption & %
\begin{minipage}[t]{0.5\columnwidth}%
\begin{spacing}{0.5}
\noindent $p\wedge\left(p\vee q\right)\equiv p$

\noindent $p\vee\left(p\wedge q\right)\equiv p$\end{spacing}
%
\end{minipage}\tabularnewline
\hline 
\noalign{\vskip\doublerulesep}
Negation & %
\begin{minipage}[t]{0.5\columnwidth}%
\begin{spacing}{0.5}
\noindent $p\vee\lnot p\equiv\mathrm{T}$

\noindent $p\wedge\lnot p\equiv\mathrm{F}$\end{spacing}
%
\end{minipage}\tabularnewline
\hline 
\noalign{\vskip\doublerulesep}
Double Negation & %
\begin{minipage}[t]{0.5\columnwidth}%
\begin{spacing}{0.5}
\noindent $\lnot\left(\lnot p\right)\equiv p$\end{spacing}
%
\end{minipage}\tabularnewline
\hline 
\end{tabular}


\subsubsection*{Involving Biconditionals}

\begin{spacing}{0.5}
\noindent $p\leftrightarrow q\equiv\left(p\rightarrow q\right)\wedge\left(q\rightarrow p\right)$

\noindent $p\leftrightarrow q\equiv\lnot p\leftrightarrow\lnot q$

\noindent $p\leftrightarrow q\equiv\left(p\wedge q\right)\vee\left(\lnot p\wedge\lnot q\right)$

\noindent $\lnot\left(p\leftrightarrow q\right)\equiv p\leftrightarrow\lnot q$
\end{spacing}


\subsubsection*{Involving Conditional Statements}

\begin{spacing}{0.5}
\noindent $p\rightarrow q\equiv\lnot p\vee q$
\end{spacing}

\noindent $p\rightarrow q\equiv\lnot q\rightarrow\lnot p$

\begin{spacing}{0.5}
\noindent $p\vee q\equiv\lnot p\rightarrow q$
\end{spacing}

\noindent $p\wedge q\equiv\lnot\left(p\rightarrow\lnot q\right)$

\begin{spacing}{0.5}
\noindent $\left(p\rightarrow q\right)\wedge\left(p\rightarrow r\right)\equiv p\rightarrow\left(q\wedge r\right)$

\noindent $\left(p\rightarrow r\right)\wedge\left(q\rightarrow r\right)\equiv\left(p\vee q\right)\rightarrow r$

\noindent $\left(p\rightarrow q\right)\vee\left(p\rightarrow r\right)\equiv p\rightarrow\left(q\vee r\right)$

\noindent $\left(p\rightarrow r\right)\vee\left(q\rightarrow r\right)\equiv\left(p\wedge q\right)\rightarrow r$
\end{spacing}


\subsection{Rules of Inference}

\begin{tabular}{ll}
\hline 
Modus Ponens & $\begin{array}{l}
p\rightarrow q\\
\underline{p\quad\quad}\\
q
\end{array}$\tabularnewline
\hline 
Modus Tollens & $\begin{array}{l}
\lnot q\\
\underline{p\rightarrow q}\\
\lnot p
\end{array}$\tabularnewline
\hline 
Associative & $\begin{array}{l}
\underline{\left(p\vee q\right)\vee r}\\
p\vee\left(q\vee r\right)
\end{array}$\tabularnewline
\hline 
Commutative & $\begin{array}{l}
\underline{p\wedge q}\\
q\wedge p
\end{array}$\tabularnewline
\hline 
Biconditional & $\begin{array}{l}
p\rightarrow q\\
\underline{q\rightarrow p}\\
p\leftrightarrow q
\end{array}$\tabularnewline
\hline 
Exportation & $\begin{array}{l}
\underline{\left(p\wedge q\right)\rightarrow r\;\;}\\
p\rightarrow\left(q\rightarrow r\right)
\end{array}$\tabularnewline
\hline 
Contraposition & $\begin{array}{l}
\underline{p\rightarrow q\quad\;}\\
\lnot q\rightarrow\lnot p
\end{array}$\tabularnewline
\hline 
Hypothetical Syllogism & $\begin{array}{l}
p\rightarrow q\\
\underline{q\rightarrow r}\\
p\rightarrow r
\end{array}$\tabularnewline
\hline 
Material Implication & $\begin{array}{l}
\underline{p\rightarrow q\;}\\
\lnot p\vee q
\end{array}$\tabularnewline
\hline 
Distributive & $\begin{array}{l}
\underline{\left(p\vee q\right)\wedge r\quad\quad}\;\\
\left(p\wedge r\right)\vee\left(q\wedge r\right)
\end{array}$\tabularnewline
\hline 
Absorption & $\begin{array}{l}
\underline{p\rightarrow q\quad\qquad}\\
p\rightarrow\left(p\wedge q\right)
\end{array}$\tabularnewline
\hline 
Disjunctive Syllogism & $\begin{array}{l}
p\vee q\\
\underline{\lnot p\quad\quad}\\
q
\end{array}$\tabularnewline
\hline 
Addition & $\begin{array}{l}
\underline{p\qquad}\\
p\vee q
\end{array}$\tabularnewline
\hline 
Simplification & $\begin{array}{l}
\underline{p\wedge q\;}\\
p
\end{array}$\tabularnewline
\hline 
Conjunction & $\begin{array}{l}
p\\
\underline{q\quad\quad}\\
p\wedge q
\end{array}$\tabularnewline
\hline 
Double Negation & $\begin{array}{l}
\underline{p\qquad}\\
\lnot\lnot p
\end{array}$\tabularnewline
\hline 
Disjunctive Simplification & $\begin{array}{l}
\underline{p\vee p\;}\\
p
\end{array}$\tabularnewline
\hline 
Resolution & $\begin{array}{l}
\;\;\:p\vee q\\
\underline{\lnot p\vee r}\\
q\vee r
\end{array}$\tabularnewline
\hline 
\end{tabular}


\subsection{Satisfiability}

A proposition is \emph{satisfiable} if some setting of the variables
makes the proposition true. For example, $p\wedge\lnot q$ is satisfiable
because the expression is true if $p$ is true or $q$ is false. On
the other hand, $p\wedge\lnot p$ is not satisfiable because the expression
as a whole is false for both settings of $p$.


\subsubsection*{2-SAT Problem}

(to follow...)


\section{Proofs}


\subsection{Mathematical Induction}

A statement $P\left(n\right)$ involving the positive integer $n$
is true for all positive integer values of $n$ is true if $P\left(1\right)$
is true and if $P\left(k\right)$ is true for any arbitrary positive
integer $k$, then $P\left(k+1\right)$ is true.

\noindent \begin{center}
\input{dominoesfalling.tikz}
\par\end{center}

The base case need not be for $n=1$. It can be adjusted to whatever
the smallest integer value $n$ assumes.


\subsection{Strong Induction}

Let $P\left(n\right)$ be a predicate defined over all integers $n,$
and let $a$ and $b$ be fixed integers with $a\le b.$ Suppose the
following two statements are true:
\begin{enumerate}
\item Base cases: $P\left(a\right),P\left(a+1\right),\ldots,P\left(b\right)$
are all true.
\item Inductive step: For any integer $k>b$, if $P\left(i\right)$ is true
for all integers $i$ with $a\le i<k$, then $P\left(k\right)$ is
true. 
\end{enumerate}
Then the statement $P\left(n\right)$ is true for all integers $n\ge a.$


\section{Recurrence Relations}


\section{Number Theory}


\subsection{Divisibility}


\subsection{Primes and Factors}


\subsection{Divisors}


\subsubsection*{Greatest Common Divisor}

This can be defined by the following recurrence relation:

\[
\gcd\left(a,b\right)=\begin{cases}
a & \text{if }b=0\\
\gcd\left(b,a\bmod b\right) & \text{else}
\end{cases}
\]



\subsection{Modular Arithmetic}


\subsubsection*{Basic Rules}

(to follow...)


\subsubsection*{Fermat's Little Theorem}

If $p$ is a prime number and $a$ is a natural number, then

\[
a^{p}\equiv a\pmod{p}
\]



\subsubsection*{Chinese Remainder Theorem}

Let $m_{1},m_{2},\ldots,m_{n}$ be pairwise relatively prime positive
integers, and $a_{1},a_{2},\ldots,a_{n}$ be arbitrary integers. Then
the system 
\[
\left\{ \begin{array}{c}
x\equiv a_{1}\pmod{m_{1}}\\
x\equiv a_{2}\pmod{m_{2}}\\
\vdots\\
x\equiv a_{n}\pmod{m_{n}}
\end{array}\right.
\]


has a unique solution modulo $m=m_{1}m_{2}\cdots m_{n}$, where $x=\sum_{k=1}^{n}a_{k}M_{k}y_{k}$,
$M_{k}=\frac{m}{m_{k}}$, and $y_{k}$ is the modular inverse of $M_{k}$
modulo $m_{k}$, i.e. $M_{k}y_{k}\equiv1\pmod{m_{k}}$.


\section{Graph Theory}


\subsection{Notation}


\subsubsection*{Fundamental Notation}

\begin{multicols*}{2}
\begin{lyxlist}{00.00.0000}
\begin{spacing}{0.5}
\item [{$G$}] \noindent graph
\item [{$V$}] \noindent vertex set
\item [{$E$}] \noindent edge set\end{spacing}

\end{lyxlist}
\end{multicols*}


\subsubsection*{Graph Invariants}

\begin{multicols}{2}
\begin{lyxlist}{00.00.0000}
\begin{spacing}{0.3}
\item [{$c\left(G\right)$}] \noindent circumference
\item [{$d\left(u,v\right)$}] \noindent distance between two vertices
\item [{$\deg\left(v\right)$}] \noindent degree of \\
a vertex
\item [{$\mathrm{gir}\left(G\right)$}] \noindent girth
\item [{$\chi\left(G\right)$}] \noindent chromatic \\
number
\item [{$\chi^{\prime}\left(G\right)$}] \noindent chromatic index
\item [{$\delta\left(G\right)$}] \noindent minimum degree
\item [{$\Delta\left(G\right)$}] \noindent maximum \\
degree
\item [{$\kappa\left(G\right)$}] \noindent vertex \\
connectivity
\item [{$\lambda\left(G\right)$}] \noindent edge \\
connectivity\end{spacing}

\end{lyxlist}
\end{multicols}

\begin{spacing}{0.3}

\subsection{\noindent Definitions}
\end{spacing}
\begin{description}
\item [{graph}] an ordered pair $\left(V,E\right)$ where $V$ is the set
of vertices and $E$ is the set of edges
\item [{simple}] a graph having neither loops nor multiple edges
\item [{multigraph}] a graph with multiple edges but no loops
\item [{pseudograph}] a graph having both loops and multiple edges
\item [{digraph}] a directed graph in which each edge has a direction
\item [{adjacency}] two distinct vertices $v$ and $w$ in a graph are
adjacent if the pair $\left\{ v,w\right\} $ is an edge
\item [{incidence}] a vertex $v$ and an edge $e$ are incident with one
another if $v\in e$
\item [{degree}] (of a vertex $v$, in symbols $\deg\left(v\right)$) the
number of vertices adjacent to $v$ 
\item [{walk}] an alternating sequence $v_{0},e_{1},v_{1},\ldots,e_{k},v_{k}$
of vertices $v_{i}$ and edges $e_{i}$ for which $e_{i}$ is incident
with $v_{i-1}$ and with $v_{i}$ for each $i$
\item [{path}] a walk whose vertices are distinct
\item [{trail}] a walk whose edges are distinct
\item [{circuit}] a trail whose first and last vertices are identical
\item [{cycle}] a circuit where each pair of whose vertices other than
the first and the last are distinct
\end{description}

\subsection{Properties}


\subsubsection*{Handshaking Lemma}

In any graph the sum of the vertex degrees is equal to twice the number
of edges.

\[
\sum_{v\in V}\deg\left(v\right)=2|E|
\]



\section{Linear Algebra}


\section{Combinatorics}


\subsection{Permutations and Combinations}


\subsubsection*{Permutation}

A permutation or ranking of $n$ objects is a listing of them in a
certain order from first to last.

The number of permutations of length $k$ from $n$ distinct objects
where repetition is not allowed is 
\[
_{n}P_{k}=\left(n\right)_{k}=\frac{n!}{\left(n-k\right)!}
\]


where $\left(n\right)_{k}$ denotes the falling factorial.


\subsubsection*{Combination}

A combination of $k$ objects taken from a collection of $n$ objects
is simply a selection of $k$ of those distinct objects without regard
to order.

The number of different combinations of $k$ objects taken from a
collection of $n$ distinct objects without repetition is

\[
_{n}C_{k}=\binom{n}{k}=\frac{n!}{k!\left(n-k\right)!}
\]



\subsection{Binomial Coefficients}

The binomial coefficient $\binom{n}{k}$ can be defined as the coefficient
of the $x^{k}$ term in the polynomial expansion of $\left(x+1\right)^{n}$,
which occurs in the binomial formula
\[
(x+y)^{n}=\sum_{k=0}^{n}{\binom{n}{k}}x^{n-k}y^{k}
\]


\[
\binom{n}{k}=\frac{n!}{k!\left(n-k\right)!}={\binom{n-1}{k-1}}+{\binom{n-1}{k}}={\binom{n}{n-k}}
\]



\subsubsection*{Pascal's Triangle}

\noindent \begin{center}
\usepgflibrary{fpu}
\def\N{10}
\tikz[x=0.75cm,y=0.5cm, 
  pascal node/.style={font=\footnotesize}, 
  row node/.style={font=\footnotesize, anchor=west, shift=(180:1)}]
  \path  
    \foreach \n in {0,...,\N} { 
      (-\N/2-1, -\n) node  [row node/.try]{Row \n:}
        \foreach \k in {0,...,\n}{
          (-\n/2+\k,-\n) node [pascal node/.try] {%
            \pgfkeys{/pgf/fpu}%
            \pgfmathparse{round(\n!/(\k!*(\n-\k)!))}%
            \pgfmathfloattoint{\pgfmathresult}%
            \pgfmathresult%
          }}};

\par\end{center}


\subsection{Generalized Permutations and Combinations}


\subsubsection*{Permutations with Repetitions}

The number of permutations of length $k$ from $n$ distinct objects
where repetition is allowed is $n^{k}$.


\subsubsection*{Permutations with Duplicate Objects}

The number of permutations of a multiset of $n$ objects made up of
$k$ distinct objects can be expressed as follows: 
\[
\binom{n}{n_{1},n_{2},\ldots,n_{k}}=\frac{n!}{n_{1}!n_{2}!\cdots n_{k}!}
\]


where $n_{i}$ represents the multiplicity of a distinct object $i$
in the multiset.


\subsubsection*{Combinations with Repetition (Stars and Bars)}

The number of combinations of length $n$ using $k$ different kinds
of objects is 
\[
_{n}R_{k}=\binom{n+k-1}{n-1}=\binom{n+k-1}{k}=\frac{\left(n+k-1\right)!}{k!\left(n-1\right)!}
\]



\paragraph*{Number of Non-negative Integer Solutions}

The number of solutions of the equation $x_{1}+x_{2}+\cdots+x_{k}=n$
in non-negative integers is $\binom{n+k-1}{k-1}$.


\paragraph*{Number of Positive Integer Solutions}

The number of solutions of the equation $x_{1}+x_{2}+\cdots+x_{k}=n$
in positive integers is $\binom{n-1}{k-1}$.


\subsection{Principle of Inclusion-Exclusion}

This provides an organized method/formula to find the number of elements
in the union of a given group of sets, the size of each set, and the
size of all possible intersections among the sets.


\subsubsection*{Two/Three Sets}

Suppose that $A,$$B,$ and $C$ are finite sets. Then:
\begin{itemize}
\item $|A\cup B|=|A|+|B|-|A\cap B|$
\item $|A\cup B\cup C|=|A|+|B|+|C|-|A\cap B|-|A\cap C|-|B\cap C|+|A\cap B\cap C|$
\end{itemize}

\subsubsection*{General Form}

For finite sets $A_{1},\ldots,A_{n}$, one has the identity:

\begin{eqnarray*}
{\biggl|}\bigcup_{{i=1}}^{n}A_{i}{\biggr|} & = & \sum_{{i=1}}^{n}\left|A_{i}\right|\;-\sum_{{1\leq i<j\leq n}}\left|A_{i}\cap A_{j}\right|\;\\
 &  & +\sum_{{1\leq i<j<k\leq n}}\left|A_{i}\cap A_{j}\cap A_{k}\right|\;\\
 &  & -\ \ldots\ +\;\left(-1\right)^{{n-1}}\left|A_{1}\cap\cdots\cap A_{n}\right|\\
 & = & \sum_{{k=1}}^{{n}}(-1)^{{k+1}}\left(\sum_{{1\leq i_{{1}}<\cdots<i_{{k}}\leq n}}\left|A_{{i_{{1}}}}\cap\cdots\cap A_{{i_{{k}}}}\right|\right)
\end{eqnarray*}



\subsection{Derangements}

A derangement is a permutation of the elements of a set, such that
no element appears in its original position. The number of derangements
of $n$ elements can be determined as follows:

\[
!n=\left(n-1\right)\left(!\left(n-1\right)+!\left(n-2\right)\right)=n!\sum_{k=0}^{n}\frac{\left(-1\right)^{k}}{k!}
\]


\textsc{oeis} \href{http://oeis.org/A000166}{A000166}: 0, 1, 2, 9,
44, 265, 1854, 14833, 133496, 1334961, 14684570, 176214841, 2290792932,
$\ldots$


\subsection{Catalan Numbers}

\begin{eqnarray*}
C_{n} & = & {\frac{1}{n+1}}{2n \choose n}={\frac{\left(2n\right)!}{\left(n+1\right)!\,n!}}=\prod\limits _{k=2}^{n}{\frac{n+k}{k}}\quad{\mbox{ for }}n\geq0\\
 & = & \binom{2n}{n}-\binom{2n}{n+1}=\sum_{i=0}^{n-1}C_{i}C_{n-i-1}
\end{eqnarray*}


\textsc{oeis} \href{http://oeis.org/A000108}{A000108}: 1, 1, 2, 5,
14, 42, 132, 429, 1430, 4862, 16796, 58786, 208012, 742900, 2674440,
9694845, 35357670, $\ldots$


\subsubsection*{Applications}
\begin{enumerate}
\begin{spacing}{0.75}
\item \noindent number of expressions containing $n$ pairs of parentheses
which are correctly matched
\item \noindent number of different ways $n+1$ factors can be completely
parenthesized
\item \noindent number of full binary trees with $n+1$ leaves
\item \noindent number of monotonic lattice paths along the edges of a grid
with $n\times n$ square cells, which do not pass above the diagonal
\item \noindent number of triangulations of a convex polygon with $n+2$
sides
\item \noindent number of permutations of $\left\{ 1,\ldots,n\right\} $
that avoid the pattern 123 (or any of the other patterns of length
3)
\item \noindent number of noncrossing partitions of the set $\left\{ 1,\ldots,n\right\} $
\item \noindent number of ways to tile a stairstep shape of height $n$
with $n$ rectangles
\item \noindent number of ways to form a \textquotedblleft mountain range\textquotedblright{}
with $n$ upstrokes and $n$ downstrokes that all stay above the original
line
\item \noindent number of semiorders on $n$ unlabeled items\end{spacing}

\end{enumerate}

\subsection{Partitions}

The function $p\left(n,k\right)$ denotes the number of ways of writing
$n$ as a sum of exactly $k$ terms.

\[
p\left(n,k\right)=\begin{cases}
1 & \text{if }n=k=0\\
0 & \text{if }n<k\\
p\left(n-1,k-1\right)+p\left(n-k,k\right) & \text{if }n\ge k
\end{cases}
\]



\subsection{Stirling Numbers}


\subsubsection*{First Kind (Cycles)}

Counts number of permutations of $n$ elements with $k$ disjoint
cycles.

\[
\left[{n\atop k}\right]=\begin{cases}
1 & \text{if }n=k=0\\
0 & \text{if }n\ne k\wedge k=0\\
\left(n-1\right)\left[{n-1\atop k}\right]+\left[{n-1\atop k-1}\right] & \text{if }n,k>0
\end{cases}
\]



\subsubsection*{Second Kind (Subsets)}

Counts the number of ways to partition a set of $n$ objects into
$k$ non-empty subsets.

\[
\left\{ {n\atop k}\right\} =\begin{cases}
1 & \text{if }n=k=0\\
0 & \text{if }n\ne k\wedge k=0\\
\left(k-1\right)\left\{ {n-1\atop k}\right\} +\left\{ {n-1\atop k-1}\right\}  & \text{if }n,k>0
\end{cases}
\]



\section{Probability}

\end{multicols*}
\end{document}
